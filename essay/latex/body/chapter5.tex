\section{总结展望}
\par{本文在python3.10环境下,基于对Github开源爬虫项目的改动实现数据爬取,调用了pandas,pyecharts,jieba,gensim等软件包,对所获得数据进行了统计分析和可视化交互展示。国内新闻传播顶刊有计算传播论文中写到利用python进行情感分析,使用snownlp进行分析,作为一个封装好的项目,再去调用已经没有难度,基础还是在于循环,判断等语句的应用。本报告并不试图去调用一些可能是用家具评论数据进行有监督机器学习的工具通过一行代码就能实现情感分析来用到自己所爬取的可能和模型预训练预料毫无关系的数据,试问这样的实现难度和简单写个函数让每一条数据调用分析后输出结果可视化有什么不同呢?背后更深的应该是整个模型实现的过程,有什么条件需要满足,能不能用到我的项目上。本报告也存在遗憾,在LDA分析的时候gensim库是做的无监督,但是确定主题数的时候没能实现一些暴力搜索之类的方法来确定主题数,还是人为尝试了很多数字之后确定的主题数,后续有时间一定解决这部分的遗憾实现编程的严谨。}
\par{写完报告主体想到读过的很多新闻传播论文,真诚希望很多文章在使用某个方法非要去赋予概念的时候能够多一点技术理性,首先理解技术逻辑、判断自己的项目是否适合使用该数据分析技术、最终分析社会问题,否则如果仅仅是冠以计算和智能的tag毫无判断也无法读懂方法核心为了让自己的研究显得更“高端”、方法更新,在用一个错误的手段分析社会解读社会,这个世界真的会因为所谓学术更好嘛?}
\par{老师对待计算机,对待编程,对待技术、对待社会的严谨,从中收获很多。希望国内计算传播越来越好!}
